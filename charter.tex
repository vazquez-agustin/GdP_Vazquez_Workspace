\documentclass[
11pt, % The default document font size, options: 10pt, 11pt, 12pt
codirector, % Uncomment to add a codirector to the title page
]{charter} 


% El títulos de la memoria, se usa en la carátula y se puede usar el cualquier lugar del documento con el comando \ttitle
\titulo{Estación meteorológica inteligente} 

% Nombre del posgrado, se usa en la carátula y se puede usar el cualquier lugar del documento con el comando \degreename
\posgrado{Carrera de Especialización en Sistemas Embebidos} 
%\posgrado{Carrera de Especialización en Internet de las Cosas} 
%\posgrado{Carrera de Especialización en Inteligencia Artificial}
%\posgrado{Maestría en Sistemas Embebidos} 
%\posgrado{Maestría en Internet de las cosas}
% IMPORTANTE: no omitir titulaciones ni tildación en los nombres, también se recomienda escribir los nombres completos (tal cual los tienen en su documento)
% Tu nombre, se puede usar el cualquier lugar del documento con el comando \authorname
\autor{Ing. Agustín Jesús Vazquez}

% El nombre del director y co-director, se puede usar el cualquier lugar del documento con el comando \supname y \cosupname y \pertesupname y \pertecosupname
\director{Ing. Maximiliano Bujaldón}
\pertenenciaDirector{EMTECH} 
\codirector{Título y Nombre del codirector} % para que aparezca en la portada se debe descomentar la opción codirector en los parámetros de documentclass
\pertenenciaCoDirector{FIUBA}

% Nombre del cliente, quien va a aprobar los resultados del proyecto, se puede usar con el comando \clientename y \empclientename
\cliente{Nombre del cliente}
\empresaCliente{Empresa del cliente}
 
\fechaINICIO{20 de agosto de 2024}		%Fecha de inicio de la cursada de GdP \fechaInicioName
\fechaFINALPlan{08 de octubre de 2024} 	%Fecha de final de cursada de GdP
\fechaFINALTrabajo{30 de junio de 2025}	%Fecha de defensa pública del trabajo final


\begin{document}

\maketitle
\thispagestyle{empty}
\pagebreak


\thispagestyle{empty}
{\setlength{\parskip}{0pt}
\tableofcontents{}
}
\pagebreak


\section*{Registros de cambios}
\label{sec:registro}


\begin{table}[ht]
\label{tab:registro}
\centering
\begin{tabularx}{\linewidth}{@{}|c|X|c|@{}}
\hline
\rowcolor[HTML]{C0C0C0} 
Revisión & \multicolumn{1}{c|}{\cellcolor[HTML]{C0C0C0}Detalles de los cambios realizados} & Fecha      \\ \hline
0      & Creación del documento                                 &\fechaInicioName \\ \hline
1      & Se completa hasta el punto 5 inclusive                & {01} de {septiembre} de 2024 \\ \hline
2      & Se completa hasta el punto 9 inclusive                & {09} de {septiembre} de 2024 \\ \hline
%3      & Se completa hasta el punto 12 inclusive                & {día} de {septiembre} de 2024 \\ \hline
%4      & Se completa el plan	                                 & {día} de {mes} de 202X \\ \hline

% Si hay más correcciones pasada la versión 4 también se deben especificar acá

\end{tabularx}
\end{table}

\pagebreak



\section*{Acta de constitución del proyecto}
\label{sec:acta}

\begin{flushright}
Buenos Aires, \fechaInicioName
\end{flushright}

\vspace{2cm}

Por medio de la presente se acuerda con el \authorname\hspace{1px} que su Trabajo Final de la \degreename\hspace{1px} se titulará ``\ttitle'' y consistirá en la implementación de un prototipo de un sistema de medición y control de parámetros medioambientales para mitigar daños en los cultivos. El trabajo tendrá un presupuesto preliminar estimado de 680 horas y un costo estimado de \textcolor{red}{\$ XXX}, con fecha de inicio el \fechaInicioName\hspace{1px} y fecha de presentación pública el \fechaFinalName.

Se adjunta a esta acta la planificación inicial.

\vfill

% Esta parte se construye sola con la información que hayan cargado en el preámbulo del documento y no debe modificarla
\begin{table}[ht]
\centering
\begin{tabular}{ccc}
\begin{tabular}[c]{@{}c@{}}Dr. Ing. Ariel Lutenberg \\ Director posgrado FIUBA\end{tabular} & \hspace{2cm} & \begin{tabular}[c]{@{}c@{}}\clientename \\ \empclientename \end{tabular} \vspace{2.5cm} \\ 
\multicolumn{3}{c}{\begin{tabular}[c]{@{}c@{}} \supname \\ Director del Trabajo Final\end{tabular}} \vspace{2.5cm} \\
\end{tabular}
\end{table}


\section{1. Descripción técnica-conceptual del proyecto a realizar}
\label{sec:descripcion}

En el contexto de la agricultura en Argentina, el monitoreo del clima se ha convertido en una rutina diaria para los productores. La incertidumbre climática es una constante que amenaza la estabilidad de las producciones agrícolas, ya sea a través de lluvias torrenciales, heladas o fuertes vientos como el zonda en Mendoza.

El presente proyecto surge en este contexto agrícola, en un país donde el clima es un factor crítico e incontrolable que afecta significativamente las actividades productivas. La motivación principal radica en la necesidad de mitigar los riesgos climáticos a los que se enfrentan los productores agrícolas. Aunque el clima no puede ser controlado, sí es posible reducir su impacto mediante una adecuada preparación y anticipación.

La propuesta consiste en diseñar e implementar una estación meteorológica que permita la medición en tiempo real de parámetros climáticos clave. Esto se logra a través de un sistema electrónico avanzado que recopila y analiza datos meteorológicos, generando alertas tempranas y proporcionando información crucial para la toma de decisiones en el campo.

Al introducir un sistema de monitoreo y control más sofisticado, se busca no solo mejorar la capacidad de respuesta ante eventos climáticos adversos, sino también fomentar la competencia en el mercado de tecnologías agrícolas, lo cual podría resultar en una disminución de precios y en un acceso más amplio a estas soluciones.

En términos de innovación, el proyecto no solo se enfoca en la recolección de datos climáticos, sino que también incorpora la predicción. A futuro, se planea la integración de inteligencia artificial para estudiar el estado de crecimiento de los cultivos, lo que representa un avance significativo respecto al estado del arte en este campo. Además, el enfoque en la adaptabilidad y personalización del sistema según las necesidades específicas de cada productor constituye un valor añadido que distingue a esta solución en el mercado.

En la \textbf{Figura 1} se presenta el diagrama de bloques del sistema. Se observa un microcontrolador, que actúa como nodo sensor del sistema, el cual recibirá información de diversos sensores, además contará con conectividad WiFi.

\begin{figure}[htpb]
\centering 
\includegraphics[width=.85\textwidth]{/home/ubuntu/Documents/Figuras/projectDiagram.png}
\caption{Diagrama en bloques del sistema.}
\label{fig:diagBloques}
\end{figure}

\vspace{25px}

\newpage

\section{2. Identificación y análisis de los interesados}
\label{sec:interesados}

\begin{table}[ht]
%\caption{Identificación de los interesados}
%\label{tab:interesados}
\begin{tabularx}{\linewidth}{@{}|l|X|X|l|@{}}
\hline
\rowcolor[HTML]{C0C0C0} 
Rol           & Nombre y Apellido & Organización 	& Puesto 	\\ \hline
Cliente       & Jurado del CESE   & FIUBA       	& Jurado del CESE       	\\ \hline
Responsable   & \authorname       & FIUBA        	& Alumno 	\\ \hline
Colaboradores & Luis Alfredo Renna&San Rafael Arcángel S.A.&Encargado general        	\\ \hline
Orientador    & \supname	      & \pertesupname 	& Director del Trabajo Final \\ \hline
Equipo        & Esp. Ing. Roberto Oscar Axt  & Transportadora de Gas del Norte S.A.  	& Especialista SCADA        	\\ \hline
Usuario final & Productores agrícolas &     -      	&     -    	\\ \hline
\end{tabularx}
\end{table}

\textbf{Colaboradores:} se tendrá en consideración la disponibilidad del encargado de San Rafael Arcángel S.A. Luis Alfredo Renna para reunir información del campo necesaria para el diseño de los sensores.

\textbf{Equipo:} se tendrá en consideración la disponibilidad del Esp. Ing. Roberto Oscar Axt para la conexión del prototipo planteado con la nube.

\section{3. Propósito del proyecto}
\label{sec:proposito}

Diseñar e implementar un prototipo de sistema electrónico para una estación de medición y control meteorológico en tiempo real. Los datos recolectados por la estación se analizarán para detectar variaciones en el clima y generar un sistema eficiente de monitoreo y alerta temprana.

\section{4. Alcance del proyecto}
\label{sec:alcance}

El proyecto incluye:
\begin{itemize}
	\item Estudio de los sensores y conexionado inalámbrico.
	\item Diseño del hardware y firmware de la estación, basado en el microcontrolador ESP32 o similar.
	\item Documentación del sistema y subsistemas.
	\item Ensayos:
	\begin{itemize}
		\item unitarios,
		\item de integración,
		\item funcionales,
		\item y de integración contínua.
		\end{itemize}
	\item Ensamblaje y prueba final.
	
\end{itemize}

\newpage
El presente proyecto no incluye:

\begin{itemize}
	\item Diseño de una aplicación web o móvil.
	\item Fabricación de PCB.
	
\end{itemize}

\section{5. Supuestos del proyecto}
\label{sec:supuestos}

Para el desarrollo del presente proyecto se supone que:

\begin{itemize}
	\item El \authorname, responsable del proyecto, cuenta con el tiempo necesario para
la planificación y ejecución del proyecto.
	\item El \authorname, responsable del proyecto, también será el auspiciante.
	\item Los componentes necesarios para el desarrollo se conseguirán en tiempo y forma.
	\item El \supname, director del proyecto, y el \authorname, responsable del
proyecto, acuerdan una reunión -semanal/quincenal/mensual- durante el desarrollo.
	\item El \authorname, responsable del proyecto, solo se ausentará de manera indefinida en caso de enfrentar una emergencia de salud grave.
\end{itemize}

\section{6. Requerimientos}
\label{sec:requerimientos}

\begin{enumerate}
	\item Requerimientos funcionales:
		\begin{enumerate}
			\item El sistema debe medir la temperatura, humedad, presión atmosférica, velocidad y dirección del viento, humedad del suelo y cantidad de lluvia.
			\item El sistema debe visualizar la información recopilada por la estación en una interfaz.
			\item El sistema debe mostrar la información recopilada por los sensores en la nube. 
		\end{enumerate}
	\item Requerimientos del sistema embebido:
		\begin{enumerate}
			\item El microcontrolador debe leer datos del sensor de temperatura, humedad y presión.
			\item El microcontrolador debe leer datos del sensor de velocidad y dirección del viento.
			\item El microcontrolador debe leer datos del sensor de humedad de suelo.
			\item El microcontrolador debe leer datos del sensor de cantidad de lluvia.			
			\item El microcontrolador recopilará datos de los sensores para su procesamiento y posterior análisis.
			\item El microcontrolador debe tener una API para comunicarse con otros sistemas.
			\item El microcontrolador debe enviar los datos a la nube si está conectado a internet.
			\item El microcontrolador debe tener un servidor para devolver datos cuando son requeridos.
		\end{enumerate}
	\newpage
	\item Requerimientos de documentación:
		\begin{enumerate}
			\item Documentación de arquitectura de software.
			\item Documentación de informes de avance.
			\item Documentación de memoria de trabajo.
		\end{enumerate}
	\item Requerimiento de testing:
		\begin{enumerate}
			\item El sistema debe probar las funcionalidad de lectura y visualización de datos.
			\item Se deben hacer pruebas de integración entre los diferentes componentes del sistema. 
			\item Se deben hacer pruebas unitarias en cada componente del sistema.
		\end{enumerate}
	\item Requerimientos de la interfaz:
		\begin{enumerate}
			\item La interfaz debe mostrar los datos en gráficos.
		\end{enumerate}
	\item Requerimientos interoperabilidad:
		\begin{enumerate}
			\item Los datos deben ser exportables en formatos estándar (CSV, JSON) para su análisis en otras plataformas.
		\end{enumerate}
\end{enumerate}

\section{7. Historias de usuarios (\textit{Product backlog})}
\label{sec:backlog}

Para la ponderación de las historias de usuario se utiliza una escala basada en la serie de
Fibonacci (0, 1, 2, 3, 5, 8, 13, 21, 34 ...) y una descripción cualitativa: bajo, medio y alto.
La estimación final del esfuerzo requerido para completar una historia surge de la sumatoria de
la ponderación anterior aplicada a las siguientes categorías: cantidad de trabajo, complejidad
y riesgo. 

El resultado de la sumatoria, de ser necesario, se aproximará al valor superior más
cercano en la escala Fibonacci.

A continuación se exponen las historias de usuario:

\begin{enumerate}

\item ``Como productor agrícola, quiero que el sistema mida la temperatura, humedad, presión atmosférica, velocidad y dirección del viento, humedad del suelo y cantidad de lluvia para poder monitorear las condiciones climáticas que afectan mis cultivos."

\textit{Story points}: 8 (complejidad: 3, dificultad: 2, incertidumbre: 3)

\item ``Como usuario, quiero que el sistema muestre la información climática recopilada en una interfaz visual para poder tomar decisiones basadas en los datos obtenidos."

\textit{Story points}: 5 (complejidad: 2, dificultad: 1, incertidumbre: 2)

\item ``Como usuario, quiero que el sistema envíe la información recopilada por los sensores a la nube para poder acceder a los datos desde cualquier dispositivo y lugar."

\textit{Story points}: 6 (complejidad: 2, dificultad: 2, incertidumbre: 2)

\end{enumerate}

\section{8. Entregables principales del proyecto}
\label{sec:entregables}

Los entregables del proyecto son:

\begin{itemize}
	\item Código fuente documentado.
	\item Manual de usuario.
	\item Diagrama de instalación.
	\item Prototipo funcional (hardware).
	\item Informe de pruebas y validaciones.
	\item Memoria del trabajo final.
\end{itemize}

\section{9. Desglose del trabajo en tareas}
\label{sec:wbs}

\begin{enumerate}

\item Diseño del sistema (160 h)
	\begin{enumerate}
	\item Diseño de la arquitectura del sistema embebido (40 h)
	\item Pruebas funcionales (16 h)
	\item Redacción de memoria de trabajo (80 h)
	\item Redacción de informes de avance (24 h)
	\end{enumerate}

\item Sensor de temperatura, humedad y presión (50 h)
	\begin{enumerate}
	\item Investigación de documentación técnica (8 h)
	\item Prueba de concepto (8 h)
	\item Desarrollo del driver (22 h)
	\item Pruebas de validación (8 h)
	\item Documentación (4 h)
	\end{enumerate}
	
\item Sensor de velocidad y dirección del viento (50 h)
	\begin{enumerate}
	\item Investigación de documentación técnica (8 h)
	\item Prueba de concepto (8 h)
	\item Desarrollo del driver (22 h)
	\item Pruebas de validación (8 h)
	\item Documentación (4 h)
	\end{enumerate}
	
\item Sensor de humedad de suelo (50 h)
	\begin{enumerate}
	\item Investigación de documentación técnica (8 h)
	\item Prueba de concepto (8 h)
	\item Desarrollo del driver (22 h)
	\item Pruebas de validación (8 h)
	\item Documentación (4 h)
	\end{enumerate}
	
\item Sensor de cantidad de lluvia (50 h)
	\begin{enumerate}
	\item Investigación de documentación técnica (8 h)
	\item Prueba de concepto (8 h)
	\item Desarrollo del driver (22 h)
	\item Pruebas de validación (8 h)
	\item Documentación (4 h)
	\end{enumerate}
	
\item Recopilación de datos (60 h)
	\begin{enumerate}
	\item Investigación sobre posibles bases de datos (14 h)
	\item Implementación de base de datos (30 h)
	\item Pruebas de validación (10 h)
	\item Documentación (6 h)
	\end{enumerate}

\item API (60 h)
	\begin{enumerate}
	\item Investigación sobre implementación de una API (12 h)
	\item Identificación de comandos para la implementación (4 h)
	\item Desarrollo de la API (30 h)
	\item Pruebas de validación (10 h)
	\item Documentación (4 h)
	\end{enumerate}

\item Servidor de datos (60 h)
	\begin{enumerate}
	\item Investigación sobre posibles servidor a implementar (16 h)	
	\item Implementación del servidor (30 h)
	\item Pruebas de validación (10 h)
	\item Documentación (4 h)
	\end{enumerate}
	
\item Gestión de datos en la nube (60 h)
	\begin{enumerate}
	\item Configuración de nube (20 h)
	\item Desarrollo de rutina de empaquetamiento y envío de datos (26 h)
	\item Pruebas de validación (10 h)
	\item Documentación (4 h)
	\end{enumerate}	

\item Interfaz de usuario (80 h)
	\begin{enumerate}
	\item Investigación sobre posibles interfaces para visualización de datos (24 h)
	\item Implementación de interfaz de visualización (48 h)
	\item Pruebas de validación (4 h)
	\item Documentación (4 h)
	\end{enumerate}		
	
\end{enumerate}

Cantidad total de horas: 680 h.

\section{10. Diagrama de Activity On Node}
\label{sec:AoN}

\begin{consigna}{red}
Armar el AoN a partir del WBS definido en la etapa anterior.

Una herramienta simple para desarrollar los diagramas es el Draw.io (\url{https://app.diagrams.net/}).
\href{https://app.diagrams.net}{Draw.io}


\begin{figure}[htpb]
\centering 
\includegraphics[width=.8\textwidth]{./Figuras/AoN.png}
\caption{Diagrama de \textit{Activity on Node}.}
\label{fig:AoN}
\end{figure}

Indicar claramente en qué unidades están expresados los tiempos.
De ser necesario indicar los caminos semi críticos y analizar sus tiempos mediante un cuadro.
Es recomendable usar colores y un cuadro indicativo describiendo qué representa cada color.

\end{consigna}

\section{11. Diagrama de Gantt}
\label{sec:gantt}

\begin{consigna}{red}
Existen muchos programas y recursos \textit{online} para hacer diagramas de Gantt, entre los cuales destacamos:

\begin{itemize}
\item Planner
\item GanttProject
\item Trello + \textit{plugins}. En el siguiente link hay un tutorial oficial: \\ \url{https://blog.trello.com/es/diagrama-de-gantt-de-un-proyecto}
\item Creately, herramienta online colaborativa. \\\url{https://creately.com/diagram/example/ieb3p3ml/LaTeX}
\item Se puede hacer en latex con el paquete \textit{pgfgantt}\\ \url{http://ctan.dcc.uchile.cl/graphics/pgf/contrib/pgfgantt/pgfgantt.pdf}
\end{itemize}

Pegar acá una captura de pantalla del diagrama de Gantt, cuidando que la letra sea suficientemente grande como para ser legible. 
Si el diagrama queda demasiado ancho, se puede pegar primero la ``tabla'' del Gantt y luego pegar la parte del diagrama de barras del diagrama de Gantt.

Configurar el software para que en la parte de la tabla muestre los códigos del EDT (WBS).\\
Configurar el software para que al lado de cada barra muestre el nombre de cada tarea.\\
Revisar que la fecha de finalización coincida con lo indicado en el Acta Constitutiva.

En la figura \ref{fig:gantt}, se muestra un ejemplo de diagrama de gantt realizado con el paquete de \textit{pgfgantt}. 
En la plantilla pueden ver el código que lo genera y usarlo de base para construir el propio.

Las fechas pueden ser calculadas utilizando alguna de las herramientas antes citadas. Sin embargo, el siguiente ejemplo
fue elaborado utilizando 
\href{https://docs.google.com/spreadsheets/d/1fBz8NhSpc4tkkhz3KjJCbh1nR_ltDkfEcZi4tZXduqs}{esta hoja de cálculo}.

Es importante destacar que el ancho del diagrama estará dado por la longitud del texto utilizado para las tareas 
(Ejemplo: tarea 1, tarea 2, etcétera) y el valor \textit{x unit}. Para mejorar la apariencia del diagrama, es necesario
ajustar este valor y, quizás, acortar los nombres de las tareas.

\begin{figure}[htpb]
  \begin{center}
    \begin{ganttchart}[
      time slot unit=day,
      time slot format=isodate,
      x unit=0.038cm,
      y unit title=0.7cm,
      y unit chart=0.6cm,
      milestone/.append style={xscale=4}
      ]{2021-03-05}{2021-12-16}
      \gantttitlecalendar*{2021-03-05}{2021-12-16}{year} \\
      \gantttitlecalendar*{2021-03-05}{2021-12-16}{month} \\
      \ganttgroup{Duración Total}{2021-03-05}{2021-12-16} \\
      %%%%%%%%%%%%%%%%%Organización
      \ganttgroup{Organización}{2021-03-05}{2021-04-16} \\
      \ganttbar{Planificación del proyecto}{2021-03-05}{2021-04-15} \\
      %%%%%%%%%%%%%%%%%Ejecución
      \ganttgroup{Ejecución}{2021-04-16}{2021-10-21} \\
      \ganttbar{Tarea 1}{2021-04-16}{2021-04-29} \\
      \ganttbar{Tarea 2}{2021-04-30}{2021-05-13} \\
      \ganttbar{Tarea 3}{2021-05-14}{2021-05-27} \\
      \ganttbar{Tarea 4}{2021-05-28}{2021-07-12} \\
      \ganttbar{Tarea 5}{2021-07-13}{2021-08-09} \\
      \ganttbar{Tarea 6}{2021-08-10}{2021-09-23} \\
      \ganttbar{Tarea 7}{2021-09-24}{2021-09-30} \\
      \ganttbar{Tarea 8}{2021-10-01}{2021-10-14} \\
      \ganttbar{Tarea 9}{2021-10-15}{2021-10-21} \\
      % %%%%%%%%%%%%%%%%%Finalización
      \ganttgroup{Finalización}{2021-10-22}{2021-12-16} \\
      \ganttbar{Memoria v1}{2021-10-22}{2021-11-04} \\
      \ganttbar{Memoria v2}{2021-11-05}{2021-11-18} \\
      \ganttbar{Memoria final}{2021-11-19}{2021-12-02} \\
      % La fecha del siguiente milestone es la fecha en que terminamos la memoria
      \ganttmilestone{Enviar memoria al director}{2021-12-02} \\
      \ganttbar{Elaborar la presentación}{2021-12-03}{2021-12-16} \\
      \ganttmilestone{Ensayo de la presentación}{2021-12-16} \\
      %%%%%%%%%%%%%%%%%%%%%%%%%%%%%%%%%%%%%%%%%%%%%%%%%%%%%%%%%%%%%%%
    \end{ganttchart}
  \end{center}
  \caption{Diagrama de gantt de ejemplo}
  \label{fig:gantt}
\end{figure}


\begin{landscape}
\begin{figure}[htpb]
\centering 
\includegraphics[height=.85\textheight]{./Figuras/Gantt-2.png}
\caption{Ejemplo de diagrama de Gantt (apaisado).} %Modificar este título acorde.
\label{fig:diagGantt}
\end{figure}

\end{landscape}

\end{consigna}


\section{12. Presupuesto detallado del proyecto}
\label{sec:presupuesto}

\begin{consigna}{red}
Si el proyecto es complejo entonces separarlo en partes:
\begin{itemize}
	\item Un total global, indicando el subtotal acumulado por cada una de las áreas.
	\item El desglose detallado del subtotal de cada una de las áreas.
\end{itemize}

IMPORTANTE: No olvidarse de considerar los COSTOS INDIRECTOS.

Incluir la aclaración de si se emplea como moneda el peso argentino (ARS) o si se usa moneda extranjera (USD, EUR, etc). Si es en moneda extranjera se debe indicar la tasa de conversión respecto a la moneda local en una fecha dada.

\end{consigna}

\begin{table}[htpb]
\centering
\begin{tabularx}{\linewidth}{@{}|X|c|r|r|@{}}
\hline
\rowcolor[HTML]{C0C0C0} 
\multicolumn{4}{|c|}{\cellcolor[HTML]{C0C0C0}COSTOS DIRECTOS} \\ \hline
\rowcolor[HTML]{C0C0C0} 
Descripción &
  \multicolumn{1}{c|}{\cellcolor[HTML]{C0C0C0}Cantidad} &
  \multicolumn{1}{c|}{\cellcolor[HTML]{C0C0C0}Valor unitario} &
  \multicolumn{1}{c|}{\cellcolor[HTML]{C0C0C0}Valor total} \\ \hline
 &
  \multicolumn{1}{c|}{} &
  \multicolumn{1}{c|}{} &
  \multicolumn{1}{c|}{} \\ \hline
 &
  \multicolumn{1}{c|}{} &
  \multicolumn{1}{c|}{} &
  \multicolumn{1}{c|}{} \\ \hline
\multicolumn{1}{|l|}{} &
   &
   &
   \\ \hline
\multicolumn{1}{|l|}{} &
   &
   &
   \\ \hline
\multicolumn{3}{|c|}{SUBTOTAL} &
  \multicolumn{1}{c|}{} \\ \hline
\rowcolor[HTML]{C0C0C0} 
\multicolumn{4}{|c|}{\cellcolor[HTML]{C0C0C0}COSTOS INDIRECTOS} \\ \hline
\rowcolor[HTML]{C0C0C0} 
Descripción &
  \multicolumn{1}{c|}{\cellcolor[HTML]{C0C0C0}Cantidad} &
  \multicolumn{1}{c|}{\cellcolor[HTML]{C0C0C0}Valor unitario} &
  \multicolumn{1}{c|}{\cellcolor[HTML]{C0C0C0}Valor total} \\ \hline
\multicolumn{1}{|l|}{} &
   &
   &
   \\ \hline
\multicolumn{1}{|l|}{} &
   &
   &
   \\ \hline
\multicolumn{1}{|l|}{} &
   &
   &
   \\ \hline
\multicolumn{3}{|c|}{SUBTOTAL} &
  \multicolumn{1}{c|}{} \\ \hline
\rowcolor[HTML]{C0C0C0}
\multicolumn{3}{|c|}{TOTAL} &
   \\ \hline
\end{tabularx}%
\end{table}


\section{13. Gestión de riesgos}
\label{sec:riesgos}

\begin{consigna}{red}
a) Identificación de los riesgos (al menos cinco) y estimación de sus consecuencias:
 
Riesgo 1: detallar el riesgo (riesgo es algo que si ocurre altera los planes previstos de forma negativa)
\begin{itemize}
	\item Severidad (S): mientras más severo, más alto es el número (usar números del 1 al 10).\\
	Justificar el motivo por el cual se asigna determinado número de severidad (S).
	\item Probabilidad de ocurrencia (O): mientras más probable, más alto es el número (usar del 1 al 10).\\
	Justificar el motivo por el cual se asigna determinado número de (O). 
\end{itemize}   

Riesgo 2:
\begin{itemize}
	\item Severidad (S): X.\\
	Justificación...
	\item Ocurrencia (O): Y.\\
	Justificación...
\end{itemize}

Riesgo 3:
\begin{itemize}
	\item Severidad (S):  X.\\
	Justificación...
	\item Ocurrencia (O): Y.\\
	Justificación...
\end{itemize}


b) Tabla de gestión de riesgos:      (El RPN se calcula como RPN=SxO)

\begin{table}[htpb]
\centering
\begin{tabularx}{\linewidth}{@{}|X|c|c|c|c|c|c|@{}}
\hline
\rowcolor[HTML]{C0C0C0} 
Riesgo & S & O & RPN & S* & O* & RPN* \\ \hline
       &   &   &     &    &    &      \\ \hline
       &   &   &     &    &    &      \\ \hline
       &   &   &     &    &    &      \\ \hline
       &   &   &     &    &    &      \\ \hline
       &   &   &     &    &    &      \\ \hline
\end{tabularx}%
\end{table}

Criterio adoptado: 

Se tomarán medidas de mitigación en los riesgos cuyos números de RPN sean mayores a...

Nota: los valores marcados con (*) en la tabla corresponden luego de haber aplicado la mitigación.

c) Plan de mitigación de los riesgos que originalmente excedían el RPN máximo establecido:
 
Riesgo 1: plan de mitigación (si por el RPN fuera necesario elaborar un plan de mitigación).
  Nueva asignación de S y O, con su respectiva justificación:
  \begin{itemize}
	\item Severidad (S*): mientras más severo, más alto es el número (usar números del 1 al 10).
          Justificar el motivo por el cual se asigna determinado número de severidad (S).
	\item Probabilidad de ocurrencia (O*): mientras más probable, más alto es el número (usar del 1 al 10).
          Justificar el motivo por el cual se asigna determinado número de (O).
	\end{itemize}

Riesgo 2: plan de mitigación (si por el RPN fuera necesario elaborar un plan de mitigación).
 
Riesgo 3: plan de mitigación (si por el RPN fuera necesario elaborar un plan de mitigación).

\end{consigna}


\section{14. Gestión de la calidad}
\label{sec:calidad}

\begin{consigna}{red}
Elija al menos diez requerimientos que a su criterio sean los más importantes/críticos/que aportan más valor y para cada uno de ellos indique las acciones de verificación y validación que permitan asegurar su cumplimiento.

\begin{itemize} 
\item Req \#1: copiar acá el requerimiento con su correspondiente número.

\begin{itemize}
	\item Verificación para confirmar si se cumplió con lo requerido antes de mostrar el sistema al cliente. Detallar.
	\item Validación con el cliente para confirmar que está de acuerdo en que se cumplió con lo requerido. Detallar. 
\end{itemize}

\end{itemize}

Tener en cuenta que en este contexto se pueden mencionar simulaciones, cálculos, revisión de hojas de datos, consulta con expertos, mediciones, etc.  

Las acciones de verificación suelen considerar al entregable como ``caja blanca'', es decir se conoce en profundidad su funcionamiento interno.  

En cambio, las acciones de validación suelen considerar al entregable como ``caja negra'', es decir, que no se conocen los detalles de su funcionamiento interno.

\end{consigna}

\section{15. Procesos de cierre}    
\label{sec:cierre}

\begin{consigna}{red}
Establecer las pautas de trabajo para realizar una reunión final de evaluación del proyecto, tal que contemple las siguientes actividades:

\begin{itemize}
	\item Pautas de trabajo que se seguirán para analizar si se respetó el Plan de Proyecto original:\\
	 - Indicar quién se ocupará de hacer esto y cuál será el procedimiento a aplicar. 
	\item Identificación de las técnicas y procedimientos útiles e inútiles que se emplearon, los problemas que surgieron y cómo se solucionaron:\\
	 - Indicar quién se ocupará de hacer esto y cuál será el procedimiento para dejar registro.
	\item Indicar quién organizará el acto de agradecimiento a todos los interesados, y en especial al equipo de trabajo y colaboradores:\\
	  - Indicar esto y quién financiará los gastos correspondientes.
\end{itemize}

\end{consigna}

\end{document}