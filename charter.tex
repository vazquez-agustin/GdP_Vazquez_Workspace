\documentclass[
11pt, % The default document font size, options: 10pt, 11pt, 12pt
codirector, % Uncomment to add a codirector to the title page
]{charter} 


% El títulos de la memoria, se usa en la carátula y se puede usar el cualquier lugar del documento con el comando \ttitle
\titulo{Estación meteorológica inteligente} 

% Nombre del posgrado, se usa en la carátula y se puede usar el cualquier lugar del documento con el comando \degreename
\posgrado{Carrera de Especialización en Sistemas Embebidos} 
%\posgrado{Carrera de Especialización en Internet de las Cosas} 
%\posgrado{Carrera de Especialización en Inteligencia Artificial}
%\posgrado{Maestría en Sistemas Embebidos} 
%\posgrado{Maestría en Internet de las cosas}
% IMPORTANTE: no omitir titulaciones ni tildación en los nombres, también se recomienda escribir los nombres completos (tal cual los tienen en su documento)
% Tu nombre, se puede usar el cualquier lugar del documento con el comando \authorname
\autor{Ing. Agustín Jesús Vazquez}

% El nombre del director y co-director, se puede usar el cualquier lugar del documento con el comando \supname y \cosupname y \pertesupname y \pertecosupname
\director{Ing. Maximiliano Bujaldón}
\pertenenciaDirector{EMTECH} 
\codirector{Título y Nombre del codirector} % para que aparezca en la portada se debe descomentar la opción codirector en los parámetros de documentclass
\pertenenciaCoDirector{FIUBA}

% Nombre del cliente, quien va a aprobar los resultados del proyecto, se puede usar con el comando \clientename y \empclientename
\cliente{Jurado CESE}
\empresaCliente{FIUBA}
 
\fechaINICIO{20 de agosto de 2024}		%Fecha de inicio de la cursada de GdP \fechaInicioName
\fechaFINALPlan{08 de octubre de 2024} 	%Fecha de final de cursada de GdP
\fechaFINALTrabajo{30 de junio de 2025}	%Fecha de defensa pública del trabajo final


\begin{document}

\maketitle
\thispagestyle{empty}
\pagebreak


\thispagestyle{empty}
{\setlength{\parskip}{0pt}
\tableofcontents{}
}
\pagebreak


\section*{Registros de cambios}
\label{sec:registro}


\begin{table}[ht]
\label{tab:registro}
\centering
\begin{tabularx}{\linewidth}{@{}|c|X|c|@{}}
\hline
\rowcolor[HTML]{C0C0C0} 
Revisión & \multicolumn{1}{c|}{\cellcolor[HTML]{C0C0C0}Detalles de los cambios realizados} & Fecha      \\ \hline
0      & Creación del documento                                 &\fechaInicioName \\ \hline
1      & Se completa hasta el punto 5 inclusive                & {01} de {septiembre} de 2024 \\ \hline
2      & Se completa hasta el punto 9 inclusive                & {09} de {septiembre} de 2024 \\ \hline
3      & Se completa hasta el punto 12 inclusive               & {16} de {septiembre} de 2024 \\ \hline
4      & Se completa el plan	                               & {22} de {septiembre} de 2024 \\ \hline

% Si hay más correcciones pasada la versión 4 también se deben especificar acá

\end{tabularx}
\end{table}

\pagebreak



\section*{Acta de constitución del proyecto}
\label{sec:acta}

\begin{flushright}
Buenos Aires, \fechaInicioName
\end{flushright}

\vspace{2cm}

Por medio de la presente se acuerda con el \authorname\hspace{1px} que su Trabajo Final de la \degreename\hspace{1px} se titulará ``\ttitle'' y consistirá en la implementación de un prototipo de un sistema de medición y control de parámetros medioambientales para mitigar daños en los cultivos. El trabajo tendrá un presupuesto preliminar estimado de 630 horas y un costo estimado de U\$D 25 622.46, con fecha de inicio el \fechaInicioName\hspace{1px} y fecha de presentación pública el \fechaFinalName.

Se adjunta a esta acta la planificación inicial.

\vfill

% Esta parte se construye sola con la información que hayan cargado en el preámbulo del documento y no debe modificarla
\begin{table}[ht]
\centering
\begin{tabular}{ccc}
\begin{tabular}[c]{@{}c@{}}Dr. Ing. Ariel Lutenberg \\ Director posgrado FIUBA\end{tabular} & \hspace{2cm} & \begin{tabular}[c]{@{}c@{}}\clientename \\ \empclientename \end{tabular} \vspace{2.5cm} \\ 
\multicolumn{3}{c}{\begin{tabular}[c]{@{}c@{}} \supname \\ Director del Trabajo Final\end{tabular}} \vspace{2.5cm} \\
\end{tabular}
\end{table}


\section{1. Descripción técnica-conceptual del proyecto a realizar}
\label{sec:descripcion}

En el contexto de la agricultura en Argentina, el monitoreo del clima se ha convertido en una rutina diaria para los productores. La incertidumbre climática es una constante que amenaza la estabilidad de las producciones agrícolas, ya sea a través de lluvias torrenciales, heladas o fuertes vientos como el zonda en Mendoza.

El presente proyecto surge en este contexto agrícola, en un país donde el clima es un factor crítico e incontrolable que afecta significativamente las actividades productivas. La motivación principal radica en la necesidad de mitigar los riesgos climáticos a los que se enfrentan los productores agrícolas. Aunque el clima no puede ser controlado, sí es posible reducir su impacto mediante una adecuada preparación y anticipación.

La propuesta consiste en diseñar e implementar una estación meteorológica que permita la medición de parámetros climáticos clave. Esto se logra a través de un sistema electrónico avanzado que recopila y analiza datos meteorológicos, generando alertas tempranas y proporcionando información crucial para la toma de decisiones en el campo.

Al introducir un sistema de monitoreo y control más sofisticado, se busca no solo mejorar la capacidad de respuesta ante eventos climáticos adversos, sino también fomentar la competencia en el mercado de tecnologías agrícolas, lo cual podría resultar en una disminución de precios y en un acceso más amplio a estas soluciones.

En términos de innovación, el proyecto no solo se enfoca en la recolección de datos climáticos, sino que también incorpora la predicción. A futuro, se planea la integración de inteligencia artificial para estudiar el estado de crecimiento de los cultivos, lo que representa un avance significativo respecto al estado del arte en este campo. Además, el enfoque en la adaptabilidad y personalización del sistema según las necesidades específicas de cada productor constituye un valor añadido que distingue a esta solución en el mercado.

En la \textbf{Figura 1} se presenta el diagrama de bloques del sistema. Se observa un microcontrolador, que actúa como nodo sensor del sistema, el cual recibirá información de diversos sensores, además contará con conectividad WiFi.

\begin{figure}[htpb]
\centering 
\includegraphics[width=.85\textwidth]{./Figuras/projectDiagram1.png}
\caption{Diagrama en bloques del sistema.}
\label{fig:diagBloques}
\end{figure}

\vspace{25px}

\newpage

\section{2. Identificación y análisis de los interesados}
\label{sec:interesados}

\begin{table}[ht]
%\caption{Identificación de los interesados}
%\label{tab:interesados}
\begin{tabularx}{\linewidth}{@{}|l|X|X|l|@{}}
\hline
\rowcolor[HTML]{C0C0C0} 
Rol           & Nombre y Apellido & Organización 	& Puesto 	\\ \hline
Cliente       & Jurado del CESE   & FIUBA       	& Jurado del CESE       	\\ \hline
Responsable   & \authorname       & FIUBA        	& Alumno 	\\ \hline
Colaboradores & Luis Alfredo Renna&San Rafael Arcángel S.A.&Encargado general        	\\ \hline
Orientador    & \supname	      & \pertesupname 	& Director del Trabajo Final \\ \hline
Equipo        & Esp. Ing. Roberto Oscar Axt  & Transportadora de Gas del Norte S.A.  	& Especialista SCADA        	\\ \hline
Usuario final & Productores agrícolas &     -      	&     -    	\\ \hline
\end{tabularx}
\end{table}

\textbf{Colaboradores:} se tendrá en consideración la disponibilidad del encargado de San Rafael Arcángel S.A. Luis Alfredo Renna para reunir información del campo necesaria para el diseño de los sensores.

\textbf{Equipo:} se tendrá en consideración la disponibilidad del Esp. Ing. Roberto Oscar Axt para la conexión del prototipo planteado con la nube.

\section{3. Propósito del proyecto}
\label{sec:proposito}

Diseñar e implementar un prototipo de sistema electrónico para una estación de medición y control meteorológico. Los datos recolectados por la estación se analizarán para detectar variaciones en el clima y generar un sistema eficiente de monitoreo y alerta temprana.

\section{4. Alcance del proyecto}
\label{sec:alcance}

El proyecto incluye:
\begin{itemize}
	\item Estudio de los sensores y conexionado inalámbrico.
	\item Diseño del hardware y firmware de la estación, basado en el microcontrolador ESP32 o similar.
	\item Documentación del sistema y subsistemas.
	\item Ensayos:
	\begin{itemize}
		\item unitarios,
		\item de integración,
		\item funcionales,
		\item y de integración contínua.
		\end{itemize}
	\item Ensamblaje y prueba final.
	
\end{itemize}

\newpage
El presente proyecto no incluye:

\begin{itemize}
	\item Diseño de una aplicación web o móvil.
	\item Fabricación de PCB.
	
\end{itemize}

\section{5. Supuestos del proyecto}
\label{sec:supuestos}

Para el desarrollo del presente proyecto se supone que:

\begin{itemize}
	\item El \authorname, responsable del proyecto, cuenta con el tiempo necesario para
la planificación y ejecución del proyecto.
	\item El \authorname, responsable del proyecto, también será el auspiciante.
	\item Los componentes necesarios para el desarrollo se conseguirán en tiempo y forma.
	\item El \supname, director del proyecto, y el \authorname, responsable del
proyecto, acuerdan una reunión -semanal/quincenal/mensual- durante el desarrollo.
	\item El \authorname, responsable del proyecto, solo se ausentará de manera indefinida en caso de enfrentar una emergencia de salud grave.
\end{itemize}

\section{6. Requerimientos}
\label{sec:requerimientos}

\begin{enumerate}
	\item Requerimientos funcionales:
		\begin{enumerate}
			\item El sistema debe medir la temperatura, humedad, presión atmosférica, velocidad y dirección del viento y humedad del suelo.
			\item El sistema debe visualizar la información recopilada por la estación en una interfaz.
			\item El sistema debe mostrar la información recopilada por los sensores en la nube. 
		\end{enumerate}
	\item Requerimientos del sistema embebido:
		\begin{enumerate}
			\item El microcontrolador debe leer datos del sensor de temperatura, humedad y presión con un margen de error de ±1°C, ±3\% y ±2hPa respectivamente.
			\item El microcontrolador debe leer datos del sensor de velocidad y dirección del viento con un margen de error de ±0.5 m/s y ±10° respectivamente.
			\item El microcontrolador debe leer datos del sensor de humedad de suelo con un margen de error de ±5\%.	
			\item El microcontrolador recopilará datos de los sensores para su procesamiento y posterior análisis.
			\item El microcontrolador debe tener una API para comunicarse con otros sistemas.
			\item El microcontrolador debe enviar los datos a la nube si está conectado a internet. La transmisión debe ocurrir al menos cada 5 minutos cuando haya conexión.
			\item El microcontrolador debe tener un servidor para devolver datos cuando son requeridos.
		\end{enumerate}
	\newpage
	\item Requerimientos de documentación:
		\begin{enumerate}
			\item Documentación de arquitectura de software.
			\item Documentación de informes de avance.
			\item Documentación de memoria de trabajo.
		\end{enumerate}
	\item Requerimiento de testing:
		\begin{enumerate}
			\item El sistema debe probar las funcionalidad de lectura y visualización de datos.
			\item Se deben hacer pruebas de integración entre los diferentes componentes del sistema. 
			\item Se deben hacer pruebas unitarias en cada componente del sistema.
		\end{enumerate}
	\item Requerimientos de la interfaz:
		\begin{enumerate}
			\item La interfaz debe mostrar los datos en gráficos y deben actualizarse al menos cada 2 minutos con nuevos datos.
		\end{enumerate}
	\item Requerimientos interoperabilidad:
		\begin{enumerate}
			\item Los datos deben ser exportables en formatos estándar (CSV, JSON) para su análisis en otras plataformas.
		\end{enumerate}
\end{enumerate}

\section{7. Historias de usuarios (\textit{Product backlog})}
\label{sec:backlog}

Para la ponderación de las historias de usuario se utiliza una escala basada en la serie de
Fibonacci (0, 1, 2, 3, 5, 8, 13, 21, 34 ...) y una descripción cualitativa: bajo, medio y alto.
La estimación final del esfuerzo requerido para completar una historia surge de la sumatoria de
la ponderación anterior aplicada a las siguientes categorías: cantidad de trabajo, complejidad
y riesgo. 

El resultado de la sumatoria, de ser necesario, se aproximará al valor superior más
cercano en la escala Fibonacci.

A continuación se exponen las historias de usuario:

\begin{enumerate}

\item ``Como productor agrícola, quiero que el sistema mida la temperatura, humedad, presión atmosférica, velocidad y dirección del viento y humedad del suelo para poder monitorear las condiciones climáticas que afectan mis cultivos."

\textit{Story points}: 8 (complejidad: 3, dificultad: 2, incertidumbre: 3)

\item ``Como usuario, quiero que el sistema muestre la información climática recopilada en una interfaz visual para poder tomar decisiones basadas en los datos obtenidos."

\textit{Story points}: 5 (complejidad: 2, dificultad: 1, incertidumbre: 2)

\item ``Como usuario, quiero que el sistema envíe la información recopilada por los sensores a la nube para poder acceder a los datos desde cualquier dispositivo y lugar."

\textit{Story points}: 6 (complejidad: 2, dificultad: 2, incertidumbre: 2)

\end{enumerate}

\section{8. Entregables principales del proyecto}
\label{sec:entregables}

Los entregables del proyecto son:

\begin{itemize}
	\item Código fuente documentado.
	\item Manual de usuario.
	\item Diagrama de instalación.
	\item Prototipo funcional (hardware).
	\item Informe de pruebas y validaciones.
	\item Memoria del trabajo final.
\end{itemize}

\section{9. Desglose del trabajo en tareas}
\label{sec:wbs}

\begin{enumerate}

\item Diseño del sistema (80 h)
	\begin{enumerate}
	\item Diseño de la arquitectura del sistema embebido (40 h)
	\item Pruebas funcionales (16 h)
	\item Redacción de informes de avance (24 h)
	\end{enumerate}

\item Sensor de temperatura, humedad y presión (50 h)
	\begin{enumerate}
	\item Investigación de documentación técnica (8 h)
	\item Prueba de concepto (8 h)
	\item Desarrollo del driver (22 h)
	\item Pruebas de validación (8 h)
	\item Documentación (4 h)
	\end{enumerate}
	
\item Sensor de velocidad y dirección del viento (50 h)
	\begin{enumerate}
	\item Investigación de documentación técnica (8 h)
	\item Prueba de concepto (8 h)
	\item Desarrollo del driver (22 h)
	\item Pruebas de validación (8 h)
	\item Documentación (4 h)
	\end{enumerate}
	
\item Sensor de humedad de suelo (50 h)
	\begin{enumerate}
	\item Investigación de documentación técnica (8 h)
	\item Prueba de concepto (8 h)
	\item Desarrollo del driver (22 h)
	\item Pruebas de validación (8 h)
	\item Documentación (4 h)
	\end{enumerate}
\newpage
\item Recopilación de datos (60 h)
	\begin{enumerate}
	\item Investigación sobre posibles bases de datos (14 h)
	\item Implementación de base de datos (30 h)
	\item Pruebas de validación (10 h)
	\item Documentación (6 h)
	\end{enumerate}

\item API (60 h)
	\begin{enumerate}
	\item Investigación sobre implementación de una API (12 h)
	\item Identificación de comandos para la implementación (4 h)
	\item Desarrollo de la API (30 h)
	\item Pruebas de validación (10 h)
	\item Documentación (4 h)
	\end{enumerate}

\item Servidor de datos (60 h)
	\begin{enumerate}
	\item Investigación sobre posibles servidor a implementar (16 h)	
	\item Implementación del servidor (30 h)
	\item Pruebas de validación (10 h)
	\item Documentación (4 h)
	\end{enumerate}
	
\item Gestión de datos en la nube (60 h)
	\begin{enumerate}
	\item Configuración de nube (20 h)
	\item Desarrollo de rutina de empaquetamiento y envío de datos (26 h)
	\item Pruebas de validación (10 h)
	\item Documentación (4 h)
	\end{enumerate}	

\item Interfaz de usuario (80 h)
	\begin{enumerate}
	\item Investigación sobre posibles interfaces para visualización de datos (24 h)
	\item Implementación de interfaz de visualización (48 h)
	\item Pruebas de validación (4 h)
	\item Documentación (4 h)
	\end{enumerate}		

\item Redacción de memoria de trabajo (80 h)
	\begin{enumerate}
	\item Taller A (40 h)
	\item Taller B (40 h)
	\end{enumerate}

	
\end{enumerate}

\textbf{Cantidad total de horas: 630 h.}

\newpage

\section{10. Diagrama de Activity On Node}
\label{sec:AoN}

En la \textbf{Figura 2} se detalla el diagrama de Activity on Node. El proyecto tendrá \textit{inicio} el día 14 de octubre de 2024, y con una fecha tentativa de fin el 30 de junio de 2025. Las tareas en conjunto representan 630 h, de las cuales 590 h son críticas, representadas con línea gruesa en el diagrama.

\begin{figure}[htpb]
\centering 
\includegraphics[width=.8\textwidth]{./Figuras/AoN_project.png}
\caption{Diagrama de \textit{Activity on Node}.}
\label{fig:AoN}
\end{figure}

\newpage

\section{11. Diagrama de Gantt}
\label{sec:gantt}

\begin{figure}[htpb]
\centering 
\includegraphics[width=.7\textwidth, height=.74\textheight]{./Figuras/GdP_T.png}
\caption{Tareas de \textit{Diagrama de Gantt}.}
\label{fig:ganttTasks}
\end{figure}

\begin{landscape}
\begin{figure}[htpb]
\centering 
\includegraphics[width=1.55\textwidth, height=1\textheight]{./Figuras/Gantt.png}
\caption{\textit{Diagrama de Gantt1} (Parte 1).} %Modificar este título acorde.
\label{fig:diagGantt}
\end{figure}
\end{landscape}

\begin{landscape}
\begin{figure}[htpb]
\centering 
\includegraphics[width=1.55\textwidth, height=1\textheight]{./Figuras/Gantt (1).png}
\caption{\textit{Diagrama de Gantt2} (Parte 2).} %Modificar este título acorde.
\label{fig:diagGantt}
\end{figure}
\end{landscape}

\begin{landscape}
\begin{figure}[htpb]
\centering 
\includegraphics[width=1.55\textwidth, height=1\textheight]{./Figuras/Gantt (2).png}
\caption{\textit{Diagrama de Gantt2} (Parte 3).} %Modificar este título acorde.
\label{fig:diagGantt}
\end{figure}
\end{landscape}


\section{12. Presupuesto detallado del proyecto}
\label{sec:presupuesto}

En el siguiente cuadro se muestra el detalle de los costos del proyecto expresados en dólares
americanos (U\$D).

\begin{table}[htpb]
\centering
\begin{tabularx}{\linewidth}{@{}|X|c|c|c|@{}}
\hline
\rowcolor[HTML]{C0C0C0} 
\multicolumn{4}{|c|}{\cellcolor[HTML]{C0C0C0}COSTOS DIRECTOS} \\ \hline
\rowcolor[HTML]{C0C0C0} 
Descripción &
  \cellcolor[HTML]{C0C0C0}Cantidad &
  \cellcolor[HTML]{C0C0C0}Valor unitario &
  \cellcolor[HTML]{C0C0C0}Valor total \\ \hline
Placa ESP32-C3-DEVKITC-02 & 1 & 8.00 & 8.00 \\ \hline
Fuente de energía & 1 & 9.49 & 9.49 \\ \hline
Sensor de rapidez de viento & 1 & 29.99 & 29.99 \\ \hline
Sensor de dirección de viento & 1 & 22.75 & 22.75 \\ \hline
Sensor de humedad de suelo & 1 & 5.69 & 5.69 \\ \hline
Sensor de temperatura, humedad y presión & 1 & 5.72 & 5.72 \\ \hline
Honorarios profesionales & 680 & 25.00 & 17 000.00 \\ \hline
\multicolumn{3}{|c|}{SUBTOTAL} & 17 081.64 \\ \hline
\rowcolor[HTML]{C0C0C0} 
\multicolumn{4}{|c|}{\cellcolor[HTML]{C0C0C0}COSTOS INDIRECTOS} \\ \hline
\rowcolor[HTML]{C0C0C0} 
Descripción & Cantidad & Valor unitario & Valor total \\ \hline
Se estima un 50\% de los costos directos & 1 & 8 540.82 & 8 540.82 \\ \hline
\multicolumn{3}{|c|}{SUBTOTAL} & 8 540.82 \\ \hline
\rowcolor[HTML]{C0C0C0}
\multicolumn{3}{|c|}{TOTAL} & 25 622.46 \\ \hline
\end{tabularx}
\end{table}

Al último cierre al 16/09/2024 la cotización es de \$942.0. Dando un total en pesos argentinos de \$24 136 357.32

\section{13. Gestión de riesgos}
\label{sec:riesgos}

a) Identificación de los riesgos y estimación de sus consecuencias:

A continuación se detallan cinco posibles riesgos inherentes al proyecto. Estos son evaluados según su grado de severidad y probabilidad de ocurrencia tomando valores de 1 a 10.
 
\textbf{Riesgo 1: demora en la entrega de los insumos.}
\begin{itemize}
	\item Severidad (S): 8 (ocho)
		\begin{itemize}
			\item Una demora aduanera, retrasaría el proyecto afectando la fecha de cierre.
		\end{itemize}
	\item Ocurrencia (O): 5 (cuatro)
		\begin{itemize}
			\item El responsable del proyecto se ubica en Irlanda, que está bien comunicado con los proveedores, principalmente chinos, pero no está excento de demoras.
		\end{itemize}
\end{itemize}   

\textbf{Riesgo 2: pérdida o destrucción del kit de desarrollo o de los componentes.}
\begin{itemize}
	\item Severidad (S): 7 (siete)
		\begin{itemize}
			\item Los plazos de entrega de los proveedores identificados son superiores a 4 semanas.
		\end{itemize}
	\item Ocurrencia (O): 3 (tres)
		\begin{itemize}
			\item El desarrollo del prototipo será llevado a cabo en casa del responsable del proyecto.
		\end{itemize}
\end{itemize}

\textbf{Riesgo 3: tiempo escaso para adquirir los conocimientos necesarios.}
\begin{itemize}
	\item Severidad (S): 6 (seis)
		\begin{itemize}
			\item Puede repercutir en un bajo cumplimiento de los requerimientos.
		\end{itemize}
	\item Ocurrencia (O): 3 (tres)
		\begin{itemize}
			\item Basado en experiencias previas, se afirma que se dedicará tiempo extra al estudio teórico y práctico en el transcurso del proyecto.
		\end{itemize}
\end{itemize}

\textbf{Riesgo 4: disponibilidad de tiempo del equipo de trabajo.}
\begin{itemize}
	\item Severidad (S): 9 (nueve)
		\begin{itemize}
			\item Se generarían retrasos y hasta incumplimiento total del proyecto.
		\end{itemize}
	\item Ocurrencia (O): 4 (tres)
		\begin{itemize}
			\item Debido a la carga laboral o temas de fuerza mayor podrían imposibilitar
avanzar con el proyecto según el cronograma establecido.
		\end{itemize}
\end{itemize}

\textbf{Riesgo 5: componentes defectuosos.}
\begin{itemize}
	\item Severidad (S): 9 (nueve)
		\begin{itemize}
			\item Si algún componente electrónico importado viene defectuoso, el proyecto se
retrasaría y se incurriría en gastos adicionales.
		\end{itemize}
	\item Ocurrencia (O): 3 (tres)
		\begin{itemize}
			\item La probabilidad de que un componente venga defectuoso no es alta, pero tampoco nula.
		\end{itemize}
\end{itemize}

b) Tabla de gestión de riesgos:      \textit{(El RPN se calcula como RPN=SxO)}

\begin{table}[htpb]
\centering
\begin{tabularx}{\linewidth}{@{}|X|c|c|c|c|c|c|@{}}
\hline
\rowcolor[HTML]{C0C0C0} 
Riesgo & S & O & RPN & S* & O* & RPN* \\ \hline
Demora en la entrega de los insumos                            &8  &5  &40   &8   &3   &24    \\ \hline
Pérdida o rotura del kit de desarrollo                         &7  &3  &21   &    &    &      \\ \hline
Tiempo escaso para adquirir los conocimientos necesarios       &6  &3  &18   &    &    &      \\ \hline
Disponibilidad de tiempo del equipo de trabajo                 &9  &4  &36   &9   &2   &18    \\ \hline
Componentes defectuosos                                        &9  &3  &27   &    &    &      \\ \hline
\end{tabularx}%
\end{table}

\textit{\textbf{Criterio adoptado:}} se tomarán medidas de mitigación en los riesgos cuyos números de RPN sean mayores a 30.

\textit{\textbf{Nota:} los valores marcados con (*) en la tabla corresponden luego de haber aplicado la mitigación.}

c) Plan de mitigación de los riesgos que originalmente excedían el RPN máximo establecido:
 
\textbf{Riesgo 1: demora en la entrega de los insumos.}\\
\textbf{Plan de mitigación:} realizar las compras con mayor anticipación y seleccionando insumos que se consigan en el mercado local y/o buscando dos o más alternativas de importación.
  \begin{itemize}
	\item Severidad (S*): 8 (ocho)
        \begin{itemize}
			\item La severidad permanece invariable.
		\end{itemize}
	\item Ocurrencia (O*): 3 (tres)
        \begin{itemize}
			\item Anticiparse al inicio del proyecto con la compra de componentes y tener alternativas no afectaría a los plazos de entrega propuestos inicialmente.
		\end{itemize}
	\end{itemize}

\textbf{Riesgo 4: disponibilidad de tiempo del equipo de trabajo.}\\
\textbf{Plan de mitigación:} realizar las consultas por escrito al director, pero sin dejar de avanzar con el resto de tareas, ya que no son dependientes entre sí. 
\begin{itemize}
	\item Severidad (S*): 9 (nueve)
        \begin{itemize}
			\item La severidad permanece invariable.
		\end{itemize}
	\item Ocurrencia (O*): 2 (dos)
        \begin{itemize}
			\item Anticipar el inicio de otras tareas en caso de una demora de respuesta por parte del director, ahorraría los tiempos muertos al avanzar en otras tareas propuestas en el cronograma.
		\end{itemize}
	\end{itemize}

\section{14. Gestión de la calidad}
\label{sec:calidad}

\begin{itemize}
	\item \underline{Requerimiento \#1.1}: el sistema debe medir la temperatura, humedad, presión atmosférica, velocidad y dirección del viento, humedad del suelo y cantidad de lluvia.
		\begin{itemize}
			\item \textbf{Verificación:} se verificarán las especificaciones técnicas de los sensores, comparando los resultados con instrumentos certificados y probando la consistencia de las mediciones.
			\item \textbf{Validación:} se contrastarán las mediciones en campo con referencias estándar, y el cliente validará la precisión y consistencia en diversas condiciones.
		\end{itemize}
			
	\item \underline{Requerimiento \#1.2}: el sistema debe mostrar la información recopilada por los sensores en la nube.
		\begin{itemize}
			\item \textbf{Verificación:} se confirmará que los datos se visualizan correctamente en la interfaz, actualizándose de forma adecuada y accesible desde distintos dispositivos.
			\item \textbf{Validación:} el cliente revisará la presentación y claridad de los datos en la interfaz y validará su funcionalidad visual.
		\end{itemize}
			
	\item \underline{Requerimiento \#1.3}: el sistema debe visualizar la información recopilada por la estación en una interfaz.
		\begin{itemize}
			\item \textbf{Verificación:} se probará la correcta transmisión y almacenamiento de los datos en la nube, con mecanismos de recuperación en caso de fallas de red.
			\item \textbf{Validación:} el cliente accederá a los datos desde la nube, confirmando su precisión y disponibilidad desde cualquier ubicación.
		\end{itemize}

	\item \underline{Requerimiento \#2.6}: el microcontrolador debe tener una API para comunicarse con otros sistemas.
		\begin{itemize}
			\item \textbf{Verificación:} se  revisará la documentación de la API y se realizarán pruebas funcionales de conexión con otros sistemas, validando los endpoints y los métodos de comunicación disponibles. Se probará la respuesta del microcontrolador ante solicitudes de datos y su capacidad para enviar información correctamente.
			\item \textbf{Validación:} el cliente realizará pruebas de integración con sus sistemas, confirmando que la API permite una comunicación fluida y eficiente entre el microcontrolador y otros sistemas externos, cumpliendo con los estándares acordados.
		\end{itemize}
		
	\item \underline{Requerimiento \#2.8}: el microcontrolador debe tener un servidor para devolver datos cuando son requeridos.
		\begin{itemize}
			\item \textbf{Verificación:} se probará la implementación del servidor en el microcontrolador, verificando que responde correctamente a las solicitudes de datos mediante protocolos estándar (HTTP, MQTT). Se realizarán pruebas de carga para asegurar que el servidor puede manejar múltiples solicitudes simultáneas sin pérdida de información.
			\item \textbf{Validación:} el  cliente realizará pruebas de consulta de datos a través del servidor, verificando que los datos solicitados sean devueltos correctamente y en el formato esperado. Se confirmará que la latencia y el rendimiento cumplen con las especificaciones requeridas.
		\end{itemize}
		
	\item \underline{Requerimiento \#3.1}: documentación de arquitectura de software.
		\begin{itemize}
			\item \textbf{Verificación:} se revisará la completitud y exactitud de la documentación técnica y manuales, asegurando que cubran todos los aspectos del sistema.
			\item \textbf{Validación:} el cliente revisará la documentación y confirmará que es clara y suficiente para operar y comprender el sistema.
		\end{itemize}
		
	\item \underline{Requerimiento \#5.1}: la interfaz debe mostrar los datos en gráficos y deben actualizarse al menos cada 2 minutos con nuevos datos.
		\begin{itemize}
			\item \textbf{Verificación:} se probará la correcta visualización de los datos en gráficos, verificando que las actualizaciones y personalizaciones funcionen adecuadamente.
			\item \textbf{Validación:} el cliente validará la claridad de los gráficos y confirmará que cumplen con sus expectativas visuales y funcionales.
		\end{itemize}

\end{itemize}

\section{15. Procesos de cierre}    
\label{sec:cierre}

Al finalizar el proyecto se establecen las siguientes actividades:

\begin{enumerate}
	\item Contrastar el desarrollo del proyecto respecto a la planificación original.
		\begin{itemize}
			\item Se llevará a cabo una comparación detallada entre el cronograma original y el real, así como entre los costos previstos y los finales.
			\item El responsable del proyecto generará una hoja de cálculo que contenga los tiempos y costos presupuestados frente a los realizados, destacando las variaciones significativas.
			\item Se evaluarán también los riesgos previstos al comienzo del proyecto y cómo se gestionaron durante su desarrollo.
		\end{itemize}

	\item Identificación de las técnicas y procedimientos útiles e inútiles que se utilizaron, y los
problemas que surgieron y como se solucionaron.
		\begin{itemize}
			\item Se convocará a los colabores para documentar los procedimientos y técnicas utilizadas, identificando aquellas que fueron efectivas y las que no generaron los resultados esperados.
			\item Además, se registrarán los principales problemas encontrados y las soluciones implementadas. Este análisis formará parte de las lecciones aprendidas del proyecto.
		\end{itemize}
		
	\item Archivado de toda la documentación del proyecto.
		\begin{itemize}
			\item El responsable del proyecto deberá organizar y archivar toda la información generada durante el proyecto, asegurando que esté accesible para futuras consultas.
			\item Se incluirán todos los documentos técnicos, registros de avances y cualquier otra documentación relevante.
		\end{itemize}
		
	\item Redacción de memoria final y presentación del proyecto.
		\begin{itemize}
			\item El responsable del proyecto redactará la memoria final, la cual será presentada en el marco de la Especialización en Sistemas Embebidos en la Facultad de Ingeniería de la Universidad de Buenos Aires.
			\item Todos los involucrados en el proyecto serán invitados a la presentación y se les ofrecerá un agradecimiento formal por su colaboración.
			\item Además, el responsable se ocupará de organizar esta presentación final y de coordinar la asistencia de los invitados.
		\end{itemize}

\end{enumerate}

\end{document}